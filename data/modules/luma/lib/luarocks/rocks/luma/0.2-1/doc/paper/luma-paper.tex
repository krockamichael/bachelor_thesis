\documentclass[12pt]{article}

\usepackage{sbc-template}

\usepackage{graphicx,url}

%\usepackage[brazil]{babel}   
\usepackage[latin1]{inputenc}  

     
\sloppy

\title{A PEG-based Macro System for Lua}

\author{Fabio Mascarenhas\inst{1}, S�rgio Medeiros\inst{1}}


\address{Departamento de Inform�tica -- PUC-Rio\\
  Av. Marqu�s de S�o Vicente, 225 -- Rio de Janeiro -- RJ -- Brazil
  \email{mascarenhas@acm.org, smedeiros@inf.puc-rio.br}
}

\newcommand{\href}[2]{#2 \cite{#1}}

\begin{document} 

\maketitle

\begin{abstract}
\end{abstract}
     
\begin{resumo} 
\end{resumo}


\section{Introduction}

\section{An Overview of Luma}

\emph{Luma} is a macro system for the Lua language that is heavily
inspired by Scheme's
\href{http://citeseer.ist.psu.edu/88826.html}{\texttt{define-syntax}/\texttt{syntax-rules} system}.
The Scheme macro system uses pattern matching to analyze the syntax
of a macro, and template substitution to build the code that the
macro expands to. This is a powerful yet simple system that builds
on top of Scheme's structural regularity (the use of S-expressions
for all code).

Luma also separates the expansion process in pattern matching and
template substitution phases, but as Lua source is unstructured
text the pattern and template languages have to work with text.
Luma uses \href{http://lpeg.luaforge.net}{\emph{LPEG}} for pattern
matching and \href{http://cosmo.luaforge.net}{\emph{Cosmo}} for
templates.

LPEG is a Lua implementation of the
\href{http://citeseer.ist.psu.edu/643425.html}{\emph{Parsing Expression Grammar} (PEG) formalism},
an EBNF-like formalism that can define a parser for any unambiguous
language. PEG is different from other grammatical formalisms in
that it can easily be used to define both the lexical and the
structural (syntactical) levels of a language. Cosmo is a template
language not unlike template languages used for web development,
and supports both simple textual substitution and iteration.

A Luma macro then consists of a \emph{pattern}, a \emph{template}
and optional definitions typically used to produce an environment
suitable for Cosmo from the LPEG output. The template can also be a
function that Luma calls with the captures produced by LPEG and
which returns a template for expansion. When you define a macro you
also need to supply a \emph{selector}, which Luma will use to
identify applications of that macro in the code it's expanding. A
selector can be any valid Lua identifier.

Macroexpansion is recursive: if you tell Luma to expand the macros
in a chunk of Lua code Luma will keep doing expansion until there
are no more macro applications in the code. This means that Luma
macros can expand to code that uses other macros, and even define
new macros. A macro application has the form:

\begin{verbatim}
  selector [[
      ... macro text ...
  ]]
\end{verbatim}
where the macro text can have balanced blocks of \verb!\[\[! and
\verb!\]\]!. If Luma has no macro registered with that selector it
just leaves that application alone (in effect it expands to
itself). This syntax for macro applications is not completely alien
to Lua, as most macro applications will be valid Lua code (a
function call with a long string), but also guarantees that all
macro applications are clearly delimited when mixed with regular
Lua code.

When expanding a macro Luma invokes LPEG with the macro's pattern
and the supplied text, then passes the template and captures to
Cosmo, and replaces the macro application with the text Cosmo
returns (recursively doing macroexpansion on this text).

The next section shows a complete implementation of a Luma macro,
including an example of its application.

\section{Walkthrough of a Luma Macro}

The macro example in this section is taken from
\href{http://portal.acm.org/citation.cfm?id=1132890}{a paper on Scheme macros}.
We think it is a good example because the syntax for this macro is
quite removed from normal Lua syntax, in effect the macro embeds a
domain specific language inside Lua, which compiles to efficient
Lua code via macroexpansion. Our macro defines
\emph{Deterministic Finite Automata}. A description of the
automaton is supplied as the macro text, and it expands to a
function which receives a string and tells if the automaton
recognizes this string or not. The following Lua code uses the
macro to define and use a simple automaton:

\begin{verbatim}
  require_for_syntax[[automaton]]
  local aut = automaton [[
    init: c -> more
    more: a -> more
          d -> more
          r -> finish
    finish: accept
  ]]
  print(aut("cadar"))  -- prints "true"
  print(aut("caxadr")) -- prints "false"
\end{verbatim}
You execute this code by saving it to a file and calling the Luma
driver script, \emph{luma}, on it. The driver macroexpands the file
and then supplies the expanded code to the Lua interpreter. The
\verb!require\_for\_syntax! macro requires a Lua module (in this
case the module \verb!automaton.lua!, which will define the
\emph{automaton} macro) at expansion time, so any macros defined by
the module are available when expanding the rest of the code.

An automaton has one or more states and each state has zero or more
transition rules and an optional tag that marks that state as a
final (acceptance) state. Each transition rule is a pair of a
character and a next state. The macro's syntax codifies that in an
LPEG pattern (a Lua string):

\begin{verbatim}
  local syntax = [[
    aut <- _ state+ -> build_aut
    char <- ([']{[ ]}['] / {.}) _
    rule <- (char '->' _ {name} _) -> build_rule 
    state <- ( {name} _ ':' _ (rule* -> {}) {'accept'?} _ )
              -> build_state
  ]]
\end{verbatim}
LPEG patterns extends regular PEG with captures: curly brackets
(\verb!\{\}!) around an pattern item tell LPEG to capture the text
that matched that item, and a right arrow (\verb!\-\>!) tells LPEG
to pass the captures of the pattern item to the left of the arrow
to the function to the right (or collect the captures in a Lua
table in the case of \verb!\-\>\{\}!. The functions are part of the
optional set of defitions. Luma defines a few default expressions
that can also be referenced in patterns: spaces and Lua comments
(\verb!\\\_!), Lua identifiers (\verb!name!), Lua numbers
(\verb!number!), and Lua strings (\verb!string!). The first PEG
production is always used to match the text (\verb!aut! in the case
of the pattern above).

The defitions for the \emph{automaton} macro contain the functions
referenced in the syntax:

\begin{verbatim}
  local defs = {
    build_rule = function (c, n)
      return { char = c, next = n }
    end,
    build_state = function (n, rs, accept)
      local final = tostring(accept == 'accept')
      return { name = n, rules = rs, final = final }
    end,
    build_aut = function (...)
      return { init = (...).name, states = { ... }, 
        substr = luma.gensym(), c = luma.gensym(),
        input = luma.gensym(), rest = luma.gensym() }
    end
  }
\end{verbatim}
This is regular Lua code. What these functions do is to put the
pattern captures in a form suitable for use by the template. As
Luma uses Cosmo this means tables and lists. The top-level
function, \verb!build\_aut!, builds the capture that is actually
passed to Cosmo, so it also defines any names that the code will
need to use for local variables so they won't clash with the names
supplied by the user (in effect this is manual
\href{http://portal.acm.org/citation.cfm?id=319838.319859}{macro hygiene}).

The template is a straightforward tail recursive implementation of
an automaton in Lua code:

\begin{verbatim}
  local code = [[(function ($input)
    local $substr = string.sub
    $states[=[
      local $name
    ]=]
    $states[=[
      $name = function ($rest)
        if #$rest == 0 then
          return $final
        end
        local $c = $substr($rest, 1, 1)
        $rest = $substr($rest, 2, #$rest)
        $rules[==[
          if $c == '$char' then
            return $next($rest)
          end
        ]==]
        return false
      end
    ]=]
    return $init($input)
  end)]]
\end{verbatim}
Cosmo replaces standalone \emph{\$name} by the corresponding text
in the table that Luma passes to it, and forms such as
\emph{\$name}[=[\emph{text}]=] and \emph{\$name}[==[\emph{text}]==]
by iterating over the corresponding list and using each element as
an environment to expand the text (as an example, if \emph{foo} is
a list containing the elements \emph{\{ bar = ``1'' \}},
\emph{\{ bar = ``2'' \}}, and \emph{\{ bar = ``3'' \}},
\emph{\$foo[=[\$bar]=]} becomes the string \emph{123}.

Defining the macro is a matter of telling Luma the selector and
components of the macro:

\begin{verbatim}
  luma.define("automaton", syntax, code, defs)
\end{verbatim}
As Luma uses LPEG as pattern language, macros that embed part of
the Lua syntax (creating a localized syntactical extension) can be
easily built by using a Lua parser written for LPEG, such as
\href{http://leg.luaforge.net}{\emph{Leg}}. The Luma distribution
has samples such as a class system, try/catch/finally exception
handling, list comprehensions, Ruby-like blocks, and declarative
pattern matching, all built using Luma and Leg.

\section{Other Macro Systems}

\section{Conclusion}


\bibliographystyle{sbc}
\bibliography{luma-paper}

\end{document}

